
\documentclass[11pt]{article}
\usepackage[english]{babel}
\usepackage[utf8]{inputenc}
\usepackage{fancyhdr}
\usepackage{enumitem}
\usepackage{array}
\usepackage{todonotes}
 
\pagestyle{fancy}
\fancyhf{}
\rhead{COSC480 Report} 

\lhead{Jake Norton (5695756)} 

\rfoot{\today}


\begin{document}

\noindent{\textsc{CVSS - Vulnerability Score Prediction}} \\
\noindent{
	Supervisor(s):
	David Eyers
	Veronica Liesaputra
}

\section{What are CVE and CVSS?}

\textit{The Common Vulnerabilities and Exposures (CVE) program is a dictionary or glossary of
	vulnerabilities that have been identified for specific code bases, such as software applications or
	open libraries.}[https://nvd.nist.gov/general/cve-process]

\subsection{Common Vulnerability Scoring System(CVSS)}

\textit{The Common Vulnerability Scoring System (CVSS) provides a way to capture the principal characteristics of a
	vulnerability and produce a numerical score reflecting its severity. The numerical score can then be translated
	into a qualitative representation (such as low, medium, high, and critical) to help organizations properly
	assess and prioritize their vulnerability management
	processes.}
%[https://www.first.org/cvss/#:~:text=The%20Common%20Vulnerability%20Scoring%20System,numerical%20score%20reflecting%20its%20severity.]

CVSS scoring is a high level way to break up vulnerabilities into different categories so that
organisations can choose which vulnerability to focus on first. CVSS in broken up into 3 distinct sections, base score,
temporal and environmental.

For brevity I will only show the specifics of CVSS 3.1 as this is by far the most commonly used version, even if it is
not the most recent.

\subsubsection*{Base Score}


Defines the avenues of attack that the vulnerability is open to. The more open a component is, the higher the score.

How complex the attack is so orchestrate. What are they prerequisites, how much domain knowledge/ background work in necessary, how much effort does the attacker need to invest to succeed.

The degree of priviledges the user needs to complete the attack.

If the exploit requires another human user to make the attack possible, E.g clicking a phishing link.

Defines if the attack can bleed into other security scopes. E.g access to one machine gives the ability to elevate privileges on other parts of the system.

Detemines what is the impact on the information access / disclosure to the attacker.

Refers to the integrity of the information within the component. I.e could the data have been modified by the attacker.

Refers to the impact of the attack on the availability of the component. E.g the attacker taking the component off the network, denying the users access.
%[https://www.first.org/cvss/v3.1/specification-document]
\subsubsection*{Temporal}

Temporal metrics describe the state of the exploit in relation to any further developments.
Modified Attack Vector (MAV)
Modified Attack Complexity (MAC)
Modified Privileges Required (MPR)
Modified User Interaction (MUI)
Modified Scope (MS)
Modified Confidentiality (MC)
Modified Integrity (MI)
Modified Availability (MA)
\bigskip

This could be:
\begin{itemize}

	Exploit Code Maturity -> The state of the attack itself, e.g has this exploit been pulled off in the wild or is it currently academic.

	Remidiation Level -> Broadly whether the exploit in question has been patched,

	Report Confidence -> The degree of confidence in the CVE report itself, the report may be in early stages where not all of the
	information is known.

\end{itemize}

Temporal metrics would be useful in general for a CVSS score, however NVD do not store these temporal metrics. As far as
I can tell there is no reason given for this specifically, though discourse[stack exchange] around the subject suggests that this is due
to a lack of verifiable reporting. From my perspective both remidiation level and report confidence feel like they could
have scores attributed to them, however finding verifiable reports on the exploits seen in the wild does seem more
tricky, though there are two relatively new organisations on this front, CISA(public sector) and inthewild.org(private
sector).

%https://security.stackexchange.com/questions/270257/cvss-v3-and-v3-1-missing-temporal-metrics-exploit-code-maturity-and-remediation
%https://www.cisa.gov/known-exploited-vulnerabilities-catalog
%https://inthewild.io/

\subsubsection*{Environmental}

The environment metrics are there so that the specific user can modify the metrics to suit their specific circumstances,
the general idea is that you can scale a metric higher or lower, as such I will not go into any more detail here.

\section{Motive}

\subsection{Issues with CVSS formula}

All possible scores from CVSS result in a normal distribution, which usually could be taken as a
good thing, however this is a manufactured facade. The values used for 3.1 are devised by experts,
however they end up being a collection of very random constants. However, somehow from this the
resulting distribution happens to end up looking like something we would expect.
\todo[inline]{Do we want it to be a normal distribution?}

Other strange things to notice are that over the years 2018 to 2021 there is a R2 value of 0.99,
which is something which generally doesn't happen. This points towards a curve which isn't natural.


\todo[inline]{Adding ordinal values makes no sense}

\subsection{Should we use CVSS?}
\todo[inline]{Read https://theoryof.predictable.software/articles/a-closer-look-at-cvss-scores/}

CVSS has an identity crisis. Throughout its history, when originally released it was touted as a
solution to the task of prioritising CVE remediation as well as an assessment of risk, "IT
management must identify and assess vulnerabilities across many disparate hardware and software
platforms. They (IT management) need to prioritize these vulnerabilities and remediate those that
pose the greatest risk. The Common Vulnerability Scoring System (CVSS) is an open framework that
addresses this issue"

%[Peter Mell, Karen Scarfone, and Sasha Romanosky. 2007. The common vulnerability scoring system
%(CVSS) and its applicability to federal agency systems. Retrieved from
%https://www.govinfo.gov/content/pkg/GOVPUB-C13-19c8184048f013016412405161920394/pdf/]

However, due to a lot of feedback from the community and security agencies, when FIRST released
version 3.1, the authors state "CVSS Measures Severity Not Risk".

\subsubsection*{Severity vs Risk}

The severity ideally is a measure of the impact(worst case? Or different levels?). Risk is the
likelihood of the event happening. However in CVSS this is a bit muddied as even the Base score
includes some aspects which defaults to worst case risk.

Should use some way of temporal / environmental. These are included in CVSS however, they are often
not used and it may be better to use EPSS, which gives a numerical value of the likelihood of the
event happening in 30 days based on previous history

FIRST give this definition of severity vs risk
\textit{
	CVSS Base scores (CVSS-B) represent "Technical Severity"
	Only takes into consideration the attributes of the vulnerability itself
	It is not recommended to use this alone to determine remediation priority

	Risk is often a religious topic… but…
	CVSS-BTE scores take into consideration the attributes of the…
	Base Score
	Threat associated with the vulnerability
	Environmental controls / Criticality
}

As mentioned by FIRST, the exact differences between risk and severity are strangely nebulous,
however here is a more usable definition as stated by NIST

"Risk is a measure of the extent to which an entity is threatened by a potential circumstance or event, and is
typically a function of: (i) the adverse impacts that would arise if the circumstance or event occurs; and (ii) the
likelihood of occurrence"
%NIST SP 800-30 Rev.1 Guide for Conducting Risk Assessments and the Committee on National Security
%Systems Information Assurance Glossary

There have been myriad complaints about this topic, generally due to the nature of how CVSS is often
used, especially in the US. There are many known occurences of the US government mandating the use
of CVSS base score as the primary framework used to prioritize remediation.


\end{document}
